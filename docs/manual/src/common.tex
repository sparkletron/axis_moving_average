\begin{titlepage}
  \begin{center}

  {\Huge AXIS\_MOVING\_AVERAGE}

  \vspace{25mm}

  \includegraphics[width=0.90\textwidth,height=\textheight,keepaspectratio]{img/AFRL.png}

  \vspace{25mm}

  \today

  \vspace{15mm}

  {\Large Jay Convertino}

  \end{center}
\end{titlepage}

\tableofcontents

\newpage

\section{Usage}

\subsection{Introduction}

\par
This core provides the AXIS Moving Average function. This impliments the moving average algorithm in an efficent way for FPGAs.
Efficent since it uses powers of two to calculate its weights. This formula implimented is \[\frac{X_{0} + ... X_{n}}{n}\] where n is constrained
to a power of two and X is a unsigned number. This also works as a low pass filter or a smothing algorithm.

\subsection{Dependencies}

\par
The following are the dependencies of the cores.

\begin{itemize}
  \item fusesoc 2.X
  \item iverilog (simulation)
  \item cocotb (simulation)
\end{itemize}

\input{src/fusesoc/depend_fusesoc_info.tex}

\subsection{In a Project}
\par


\section{Architecture}
\par
The only module is the axis\_moving\_average module. It is listed below.

\begin{itemize}
  \item \textbf{axis\_moving\_average} Impliment moving average algorithm (see core for documentation).
\end{itemize}

\par
This core only uses a combinatoral method to divide the accumulator. Since all weights are powers of two this is done with a part select based on bit position.

\par
The always block has the following steps.
\begin{enumerate}
\item If there is valid data, sum the new data into the accumulator and remove the top element in the buffer from the accumulator.
\item Insert the new element into the buffer.
\item Shift the buffer to so that old elements at the top of the buffer are shifted out.
\end{enumerate}

Please see \ref{Module Documentation} for more information.

\section{Building}

\par
The AXIS Moving Average core is written in Verilog 2001. They should synthesize in any modern FPGA software. The core comes as a fusesoc packaged core and can be
included in any other core. Be sure to make sure you have meet the dependencies listed in the previous section.

\subsection{fusesoc}
\par
Fusesoc is a system for building FPGA software without relying on the internal project management of the tool. Avoiding vendor lock in to Vivado or Quartus.
These cores, when included in a project, can be easily integrated and targets created based upon the end developer needs. The core by itself is not a part of
a system and should be integrated into a fusesoc based system. Simulations are setup to use fusesoc and are a part of its targets.

\subsection{Source Files}

\subsubsection{fusesoc\_info File List}
\begin{itemize}
\item src
	\begin{itemize}
	\item[$\space$] Type: verilogSource
	\item src/axis\_moving\_average.v
	\end{itemize}
\item tb
	\begin{itemize}
	\item {'tb/tb\_axis.v': {'file\_type': 'verilogSource'}}
	\end{itemize}
\end{itemize}


\subsection{Targets}

\input{src/fusesoc/targets_fusesoc_info.tex}

\subsection{Directory Guide}

\par
Below highlights important folders from the root of the directory.

\begin{enumerate}
  \item \textbf{docs} Contains all documentation related to this project.
    \begin{itemize}
      \item \textbf{manual} Contains user manual and github page that are generated from the latex sources.
    \end{itemize}
  \item \textbf{src} Contains source files for the core
  \item \textbf{tb} Contains test bench files for iverilog and cocotb
    \begin{itemize}
      \item \textbf{cocotb} testbench files
    \end{itemize}
\end{enumerate}

\newpage

\section{Simulation}
\par
There are a few different simulations that can be run for this core.

\subsection{iverilog}
\par
iverilog is used for simple test benches for quick verification, visually, of the core.

\subsection{cocotb}
\par
Future simulations will use cocotb. This feature is not yet implemented.

\newpage

\section{Module Documentation} \label{Module Documentation}

\par
There is a single async module for this core.

\begin{itemize}
\item \textbf{axis\_moving\_average} AXIS Moving Average to uP converter\\
\end{itemize}
The next sections document the module in great detail.

