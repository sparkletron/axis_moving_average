\begin{titlepage}
  \begin{center}

  {\Huge AXIS\_MOVING\_AVERAGE}

  \vspace{25mm}

  \includegraphics[width=0.90\textwidth,height=\textheight,keepaspectratio]{img/AFRL.png}

  \vspace{25mm}

  \today

  \vspace{15mm}

  {\Large Jay Convertino}

  \end{center}
\end{titlepage}

\tableofcontents

\newpage

\section{Usage}

\subsection{Introduction}

\par
This core provides the AXIS Moving Average function. This impliments the moving average algorithm in an efficent way for FPGAs.
Efficent since it uses powers of two to calculate its weights. This formula implimented is where n is constrained
to a power of two and X is a unsigned number. This also works as a low pass filter or a smothing algorithm.

\subsection{Dependencies}

\par
The following are the dependencies of the cores.

\begin{itemize}
  \item fusesoc 2.X
  \item iverilog (simulation)
  \item cocotb (simulation)
\end{itemize}

\input{src/fusesoc/depend_fusesoc_info.tex}

\subsection{In a Project}
\par
Simply use this core between a sink and source AXIS devices. This has been tested with unsigned data types. Check the code to see if others will work correctly.

\section{Architecture}
\par
The only module is the axis\_moving\_average module. This is a continous output rather than a wait and release on range. It is listed below.

\begin{itemize}
  \item \textbf{axis\_moving\_average} Implement moving average algorithm (see core for documentation).
\end{itemize}

Internally this uses the simple moving average algorithm for the output average. The weight(n) is rounded up to the nearist power of two.
\begin{equation}
  SMA = \sum_{i=0}^{\log_2 n}\frac{D_0+\dots+D_n}{n}
\end{equation}

\par
This core only uses a combinatoral method to divide the accumulator. Since all weights are powers of two this is done with a part select based on bit position.

\par
The always block has the following steps.
\begin{enumerate}
\item If there is valid data, sum the new data into the accumulator and remove the top element in the buffer from the accumulator.
\item Insert the new element into the buffer.
\item Shift the buffer to so that old elements at the top of the buffer are shifted out.
\end{enumerate}

Please see \ref{Module Documentation} for more information.

\section{Building}

\par
The AXIS Moving Average core is written in Verilog 2001. They should synthesize in any modern FPGA software. The core comes as a fusesoc packaged core and can be included in any other core. Be sure to make sure you have meet the dependencies listed in the previous section. Linting is performed by verible using the lint target.

\subsection{fusesoc}
\par
Fusesoc is a system for building FPGA software without relying on the internal project management of the tool. Avoiding vendor lock in to Vivado or Quartus.
These cores, when included in a project, can be easily integrated and targets created based upon the end developer needs. The core by itself is not a part of
a system and should be integrated into a fusesoc based system. Simulations are setup to use fusesoc and are a part of its targets.

\subsection{Source Files}

\subsubsection{fusesoc\_info File List}
\begin{itemize}
\item src
	\begin{itemize}
	\item[$\space$] Type: verilogSource
	\item src/axis\_moving\_average.v
	\end{itemize}
\item tb
	\begin{itemize}
	\item {'tb/tb\_axis.v': {'file\_type': 'verilogSource'}}
	\end{itemize}
\end{itemize}


\subsection{Targets}

\input{src/fusesoc/targets_fusesoc_info.tex}

\subsection{Directory Guide}

\par
Below highlights important folders from the root of the directory.

\begin{enumerate}
  \item \textbf{docs} Contains all documentation related to this project.
    \begin{itemize}
      \item \textbf{manual} Contains user manual and github page that are generated from the latex sources.
    \end{itemize}
  \item \textbf{src} Contains source files for the core
  \item \textbf{tb} Contains test bench files for iverilog and cocotb
\end{enumerate}

\newpage

\section{Simulation}
\par
There are a few different simulations that can be run for this core. All currently use iVerilog (icarus) to run. The first is iverilog, which
uses verilog only for the simulations. The other is cocotb. This does a unit test approach to the testing and gives a list of tests that pass
or fail.

\subsection{iverilog}
\par
All simulation targets that do NOT have cocotb in the name use a verilog test bench with verilog stimulus components. These all read in a file
and then write a file that has been processed by the data width converter. Then the input and output file are compared with a MD5 sum to check that they
match. If they do not match then the test has failed. All of these tests provide fst output files for viewing the waveform in the there
target build folder.

\subsection{cocotb}
\par
To use the cocotb tests you must install the following python libraries.
\begin{lstlisting}[language=bash]
  $ pip install cocotb
  $ pip install cocotbext-axi
\end{lstlisting}

Then you must use the cocotb sim target. In this case it is sim\_cocotb. This target can be run with various bus and fifo parameters.
\begin{lstlisting}[language=bash]
  $ fusesoc run --target sim_cocotb AFRL:math:axis_moving_average:1.0.1 --BUS_WIDTH=8 --WEIGHT=32
\end{lstlisting}

The following is an example command to run through various parameters without typing them one by one.
\begin{lstlisting}[language=bash]
  $ for i in {1..32}; do sleep 5; export RY=$(((RANDOM*2)%32)); fusesoc run --target sim_cocotb AFRL:math:axis_moving_average:1.0.1 --BUS_WIDTH=$i --WEIGHT=$RY; echo "BUS WIDTH:" $i "WEIGHT:" $RY; done
\end{lstlisting}
\newpage

\section{Code Documentation} \label{Code Documentation}

\par
Natural docs is used to generate documentation for this project. The next lists the following sections.

\begin{itemize}
\item \textbf{axis\_moving\_average} AXIS moving average core.\\
\item \textbf{tb\_axis} Verilog test bench.\\
\item \textbf{tb\_cocotb verilog} Verilog test bench base for cocotb.\\
\item \textbf{tb\_cocotb python} cocotb unit test functions.\\
\end{itemize}

